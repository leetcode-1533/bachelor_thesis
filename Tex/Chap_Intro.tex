
\chapter{矩阵计算}
\label{chap:introduction}

用户并无~\LaTeX{}~使用经验,本模板将~\LaTeX{}~的复杂性尽可能地进行了封装,开放出简单的接口,以便于使用者可以轻易地使用,同时,对使用~\LaTeX{}~撰写论文所遇到的一些主要难题,如插入图片、文献索引等,进行了详细的说明,并提供了相应的代码样本,理解了上述问题后,对于初学者而言,使用此模板撰写其学文论文将不存在实质性的困难,所以,如果您是初学者,请不要直接放弃,因为同样作为初学者的我,十分明白让~\LaTeX{}~变得简单易用的重要性,而这正是本模板所体现的.

此中国科学院大学学位论文模板~\texttt{ucasthesis}~基于吴凌云的~\texttt{CASthesis}~模板发展而来,~ucasthesis~文档类的基础架构为~ctexbook~文档类。当前~ucasthesis~ 模板满足最新的中国科学院大学学位论文撰写要求和封面设定。模板提供了~pdflatex~或~xelatex~(默认,推荐) 编译方式,完美地支持中文书签、中文渲染、中文粗体显示、拷贝~pdf~中的文本到其他文本编辑器等特性,此外,对模板的文档结构进行了精心设计,撰写了编译脚本提高模板的易用性和使用效率。

宏包的目的是简化学位论文的撰写,模板文档的默认设定是十分规范的,从而论文作者可以将精力集中到论文的内容上,而不需要在版面设置上花费精力。 同时,在编写模板的~\LaTeX{}~文档代码过程中,作者对各结构和命令进行了十分详细的注解,并提供了整洁一致的代码结构,对文档的仔细阅读可以为初学的您提供一个学习~\LaTeX{}~的窗口。除此之外,整个模板的架构十分注重通用性,事实上,本模板不仅是中国科学院大学学文论文模板,同时,也是使用~\LaTeX{}~撰写中英文~article~或~book~的通用模板,并为使用者的个性化设定提供了接口和相应的代码。

\section{系统要求}

\texttt{ucasthesis}宏包可以在目前大多数的~\TeX{}~系统中使用,例如~C\TeX{}、MiK\TeX{}、\TeX{}Live。考虑到大多数用户将是~Windows~使用者,推荐安装最新的~C\TeX{}~套装,C\TeX{}~套装中包含了本模板中出的各类宏包,用户无需额外的设置即可使用。

\texttt{ucasthesis}~宏包通过~\texttt{ctexbook}~宏包来获得中文支持。~\texttt{ctexbook}~宏包提供了一个统一的中文~\LaTeX{}~书籍文档框架。

\section{问题反馈}

\begin{center}
莫晃锐 (mohuangrui) \quad mohuangrui@gmail.com
\end{center}

欢迎大家反馈自己的使用情况,使我们可以不断改进宏包。
