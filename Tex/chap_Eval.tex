\chapter{模型评价}
\section{不同尺度下的性能}
下面将就各种特征提取算法在不同尺度下的性能做出比较. 具体的设置与之前的设置一样,然后构建高斯金字塔,探究不同的特征提取算法在不同尺度下的图像提取性能.

\section{F-Score}
以上章节仅就\textit{识别率}这一指标来评价,实际上,这是不全面的,因为没有考虑\textit{False Positive}和\textit{True Negative}的矛盾问题,具体来说,
\begin{itemize}
	\item 在搜索人脸的过程,人们通常希望系统能提出更多的候选图片,这就需要降低系统阈值,使得更多的答案能被接受.这时候,\textit{Positive}的比例就希望大一些,但这提升了\textit{False Positive}降低了\textit{True Negative}.
	\item 在较为精密的系统中,如使用人脸作为密码进行支付,人们通常希望提高系统阈值.\textit{Negative}的比例就希望大一些,但这降低了\textit{False Positive}提升了\textit{True Negative}.
\end{itemize} 
下面就本文模型中较好结果计算了它的F-Score.


\section{结论与展望}

近半年的毕业设计终于告一段落了,笔者在其中学到了不少知识,如降维算法,模型评价等知识.但是,受限于个人能力和时间投入的原因,有很多地方只得匆忙完成,不能再深入讨论下去了,本部分将可以深入探究的地方罗列一下,期望日后能得以机会加以完善.

\paragraph{全局识别的第二种架构}
本文所有的全局识别算法都是建立在以人脸图像为基本单元的基础上,由于人脸图像样本数目远小于像素的数目,这样做比较快.但是,也不可否认,存在第二种方法.在\ref{sec:arch2}中列出了以像素为基本单元的方法.目前由于运算时间的问题没能继续下去.

\paragraph{NMF算法的迭代方法研究}
本文所用的NMF迭代算法是比较早的算法,根据\cite{lin2007projected}. 本文所用的NMF算法是存在一定问题的,作者希望就NMF这一算法深入研究.

\paragraph{ICA算法应用于图片压缩}
作者在研究论文时,发现了将图片切割并使用分块的PCA或ICA来实现图像压缩的算法.以图\ref{fig:ica_base}为例,ICA算法有很强的稀疏性,因此适合以保留主要参数的方法进行图片压缩.ICA图像压缩由于不是在频谱上压缩,因此有保留图像细节的优点.

\paragraph{章节\ref{sec:pni_recon}的应用}
实际上\ref{sec:pni_recon}中MSE的差别可以被加以利用来识别场景的.可以看到,人脸压缩并恢复后的图像的与原始图像的差别小于非人脸压缩并恢复后的图像,这可以从这个角度识别人脸.

\paragraph{SIFT算法的编码方法}
本文实现的SIFT编码方法比较粗略,实际上还存在更多更加优秀的算法,受限于时间作者没能一一实现.

\paragraph{Gabor算子}
本文的局部特征提取算法只实现了SIFT算法,实际还有\textit{Gabor}等算法,作者希望有机会时把这部分内容也加进去.

\paragraph{KD树进行人脸搜索}
实际上,本项目还实现了基于距离的人脸搜索过程,但是,很遗憾的,仅仅基于距离很难搜索到相近的人.因此该部分并没有产生较好的结果,作者希望可以加以模糊逻辑等方法来优化这个过程.同时,作者指出这个方法的优越性,距离搜索可以通过KD树来完成,性能非常好,可以分布计算,非常构建人脸数据库,在大量人群中快速搜索.
