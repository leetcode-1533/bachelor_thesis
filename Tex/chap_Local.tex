\chapter{基于局部的特征提取算法}
在上一节的学习过程中,笔者渐渐感觉到计算算子对人脸识别,分类系统的影响非常大,毕竟,人脸图像被分解后,仅仅由一个很短的向量来表示,那该编码系统对样本的区分程度很大程度上就能决定后续分类系统的性能.因此,我渐渐发现了基于局部的特征提取算法,它们有些具有\textit{Content Awareness}的特性,对背景噪声有一定的区分能力,并且可以精细的比较样本的细节之处,因此可以达到精确的识别性能.本章节以SIFT算子为例做介绍.
\subsection{SIFT的原理}
本章节主要基于\cite{lowe2004distinctive, issolah2013sift, juan2009comparison, siftopencv, siftvlfeat, siftubc,lowe1999object}\newline

SIFT算子的计算方法共分4步.

\subsubsection{SIFT-PCA算法}
由于SIFT运算结果过长,笔者实现了SIFT-PCA算子\cite{ke2004pca},它是一种采用块压缩技术的基于SIFT1-3步的方法.具有可调节的步长.

\subsection{SIFT的编码方法}
由于SIFT算子是\textit{Content Aware}的,其特征提取的结果是由若干个提取位置上的描述子来描述的,因此,长度并不固定.这对后续的分类算法并不友好.因此,SIFT的结果大都被编码了,编码方法通常有以下的方法.\cite{chatfield2011devil}

\subsection{NBNN分类法}
由于SIFT的特性,本项目在上一节固定编码的同时,也结合SIFT特性实现了简单的NBNN算法.NBNN算法不需要对SIFT进行特殊的编码,同时可以并行运算.并且结果也很好.