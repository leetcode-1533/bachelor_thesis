\chapter{Introduction}

\section{Project Objective}
The initial purpose of this project is to test the performance of different \textit{Decomposition Algorithms} in terms of \textit{Facial Images}. PCA, NMF, ICA were implemented and has successfully compressed facial images with around 10,000 pixels into vectors of length 30 to 100. Besides that, a simple face recognition system was built with SVM as the classifier and the overall accuracy of successful recognition can be around 85\%. However, as the project expands, several drawbacks of those algorithms gradually revealed. 
\begin{enumerate}
    \item PCA, NMF, ICA are all so-called \textbf{global approaches}. While greatly compressed the data, those algorithms cannot preserve face crucial details. Therefore, cannot be served as an accurate description of the face. And I felt hard to further increase the recognition accuracy when it comes to very similar people.
    \item PCA, NMF, ICA functions at the global scope. And cannot effectively discriminate against the background. And cannot provide an effective solution for problems like alignment.
\end{enumerate}

\textbf{Component-based approaches} were presented to conquer those problems. In comparison to global 
approaches, it only concentrates on the areas of images with the largest changes(or with maximum energy). And can provide a more accurate description of people. SIFT were implemented, and the overall accuracy can be 99\% at its maximum. However, those algorithms are not perfect as well, for the following reasons,
\todo{Cons: Component-based approaches}
\begin{enumerate}
    \item Component-based approaches cannot effectively compress data. Taking SIFT as an example, it used 128 double numbers for a single descriptor and a facial image with around 10,000 pixels can have more than 30 descriptors, comprising of more than 3,000 double numbers. Not only so, a larger size of vectors will also drastically increase the computation difficulties.
    \item Component-based approaches are irreversible. The general facial information was lost and the original image cannot be reconstructed.
\end{enumerate}

As you will see, these doesn't exist an all-around algorithm. And this project emphasizes on highlighting those differences from the perspective of global versus component-based approaches. And to provide some insightful suggestions when it comes to real application.
\todo{Background Research}

\section{Project Scope}
\subsection{Face Recognition procedure}
A typical face recognition system usually consists of the following procedures:
\begin{enumerate}
    \item{Image Processing: } This stage treated the facial images as a normal image. And algorithms applied can be scale-transformation, anti-noising, anti-alignment, color transformation into grayscale etc.
    \item{Face Detection: } Face usually is not the only object in the frame. And this stage tries to identify the size and location of the face.
    \item{Decomposition: } Face images usually are represented by a vector with more than 10,000 elements and usually it is not plausible to process it directly. This stage tries to transform the representation of the image into other domains. And to make this representation more effective and meaningful.
    \item{Classification: } After stage 3, a facial image has been successfully transformed into a much more representative format. And it becomes a typical machine learning problem. It is to predict the label(class, identity) of the vector with or without posterior information.
\end{enumerate}
\subsection{Real Applications}
And it is important to notice that there exist different types of face image related algorithms:
\subsection{This Project}

\section{Project Structure}

\section{Potential Applications}
It is also important to stress that although this project deals solely with facial images, it actually is a practice of solving high dimension data. There exist lots similar questions which require solving massive amount of information with lots of redundancy. Examples of those questions can be:
\begin{itemize}
    \item{Genetic Data Processing: }Genes are composed of four components: A, T, C, G. And the average length of genes can be of 10-15k\cite{twine2011whole}. Decomposition and feature recognition are useful in finding the pattern of the gene.
    \item{Music Genre Recognition: }Music can be regarded as electrical waves with varying aptitude. Decomposition and feature recognition can be applied to find the genre of the music or even to recognize its singer.
\end{itemize}