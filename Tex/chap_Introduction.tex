\chapter{Introduction}

\section{Project Development Track}
The initial purpose of this project is to test the performance of difference \textit{Decomposition Algorithms} in terms of \textit{Facial Images}. PCA, NMF, ICA were implemented, and has successfully compressed facial images with around 10,000 pixels into vectors of length 30 to 100. Besides that, a simple face recognition system were built with SVM as the classifier and the overall accuracy for successfully recognition can be around 85\%. However, as the project expends, several drawbacks of those algorithms gradually revealed. 
\begin{enumerate}
    \item PCA, NMF, ICA are all so called \textbf{global approaches}. While greatly compressed the data, those algorithms cannot preserve face crucial details. Therefore cannot be served as an accurate description of the face. And I felt hard to further increase the recognition accuracy when it comes to very similar people.
    \item PCA, NMF, ICA functions at the global scope. And cannot effectively discriminate against background. And cannot provide an effective solution for problem like alignment.
\end{enumerate}

\textbf{Component-based approaches} were presented to conquer those problems. In comparison to global 
approaches, it only concentrates on the areas of image with the maximum changes(or with maximum energy). And therefore can provide a more accurate description of people. SIFT were implemented, and the overall accuracy can be 99\% at its maximum. However, those algorithms are perfect as well, for the following reasons,
\begin{enumerate}
    \item Component-based approaches cannot effectively compress the data. Taking SIFT as an example, it used 128 double number for a single descriptor and an facial image with around 10,000 pixels can have more than 30 descriptors, comprising of more than 3,000 double number.
    \item Component-based approaches are irreversible. The general facial information were lost and the original image cannot be reconstructed.
\end{enumerate}

And this project emphasize on comparing the difference between those algorithms.  
 

\section{涉及内容}
\subsection{人脸识别的三个过程}
人脸识别作为一种从图像中获取有用信息的
\section{背景调查}
\subsection{通常测试领域}

\section{本报告结构}
